\documentclass[a4paper, 11pt, reqno]{article}

\usepackage[T1]{fontenc}
\usepackage{geometry}
\usepackage{mathtools, amsmath, amssymb}

\geometry{letterpaper}

\title{Complex Identities}
\author{Satvik Saha}
\date{}

\newcounter{prob}
\def\problem{\stepcounter{prob}\paragraph{Problem \arabic{prob}}}
\def\solution{\paragraph{Solution}}

\begin{document}
	\maketitle

	\problem
	Prove the following, for $(n > 1)$.
	\begin{equation*}
		C \; = \;  \sum_{k=0}^{n-1} \cos \left( \phi + \frac{2k\pi}{n}  \right) \; = \; 0 
	\end{equation*}
	\begin{equation*}
		S \; = \;  \sum_{k=0}^{n-1} \sin \left( \phi + \frac{2k\pi}{n}  \right) \; = \; 0 
	\end{equation*}
	where $n \in \mathbb{N}$ and $\phi \in \mathbb{R}$.
	
	\solution
	Note that $e^{2\pi i} = 1$.
	Let $z$ be an $n^{\text{th}}$ root of unity, satisfying $z^{n} = 1$.
	\begin{align*}
		z^{n} - 1 \; &= \; 0	\\
		(z - 1)(z^{n-1} + z^{n-2} + \dots + 1) \; &= \; 0 
	\end{align*}
	Let $z$ be defined as $e^{2\pi i/n}$.
	Thus, $z \neq 1$,
	\begin{align}
		\sum_{k=0}^{n-1} z^{k} \; &= \; 0	\label{eq:uRootSum}
	\end{align}
	Using Euler's Formula, $e^{\varphi i} \; = \; \cos\varphi + i\sin\varphi$, we can write
	\begin{align*}
		e^{\phi i}\cdot z^{k} \; = \; e^{\phi i} \cdot  e^{2k\pi i / n}
				\; &= \;  \cos\left(\phi + \frac{2k\pi}{n}\right) + i\sin\left(\phi + \frac{2k\pi}{n}\right) \\
		e^{\phi i}\cdot \sum_{k=0}^{n-1} z^{k} \; &= \; 
				   \sum_{k=0}^{n-1} \cos\left(\phi + \frac{2k\pi}{n}\right)
				+ i\sum_{k=0}^{n-1} \sin\left(\phi + \frac{2k\pi}{n}\right) \\
	\end{align*}
	Using (\ref{eq:uRootSum}), this reduces to
	\begin{equation*}
		\boxed{0	\; = \; 	C \;+\; iS}
	\end{equation*}
	Comparing the real and imaginary parts of the above, we can conclude that $C = S = 0$.

	\problem
	Let $a + b + c = 0$. Prove the following, where $\omega^3 - 1 = 0$.
	\begin{equation*}
		(a + b\omega + c\omega ^2)^3 + (a + b\omega ^2 + c\omega)^3 = 27abc
	\end{equation*}
	
	\solution
	Note the following identities.
	\begin{align}
		1 + \omega + \omega ^2  \; &= \; 0 	\label{eq:cRoots}	\\ 
		x^3 + y^3  \; &= \;  (x + y)(x\omega^2 + y\omega)(x\omega + y\omega^2) \label{eq:cSum}
	\end{align}
	Let $x  =  a + b\omega + c\omega ^2$ and $y  =  a + b\omega ^2 + c\omega$.
	Clearly,
	\begin{align*}
		x + y \; &= \; 2a + b(\omega + \omega^2) + c (\omega^2 + \omega )	\\
			\; &= \;  2a - b - c 			\\
			\; &= \;  3a				\\
		x\omega^2 + y\omega  \; &= \; a(\omega^2 + \omega) + 2b + c(\omega + \omega^2) \\
			\; &= \;  - a + 2b + - c		\\
			\; &= \;  3b				\\
		x\omega + y\omega^2  \; &= \; a(\omega + \omega^2) + b(\omega ^2 + \omega) + 2c \\
			\; &= \;  - a - b + 2c			\\
			\; &= \;  3c				
	\end{align*}
	Using (\ref{eq:cSum}), we can write
	\begin{align*}
		x^3 + y^3 \; &= \; (3a)(3b)(3c)  \; = \; 27abc  
	\end{align*}
	which, with change in notation, is the desired result.
\end{document}
