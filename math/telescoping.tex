\documentclass[a4paper, 11pt, reqno]{article}

\usepackage[T1]{fontenc}
\usepackage{geometry}
\usepackage{mathtools, amsmath, amssymb}

\geometry{letterpaper}

\title{Telescoping Sums}
\author{Satvik Saha}
\date{}

\begin{document}
	\maketitle
	
	A sum in which subsequent terms cancel each other, leaving only its initial and final terms is called a
	\textit{telescoping sum}.
	\begin{align*}
		S  \; &= \; \sum_{k=0}^{n-1} (T_{k+1} - T_{k})	\\
		   \; &= \; (T_1 - T_0) + (T_2 - T_1) + (T_3 - T_2) + \dots (T_{n} - T_{n-1})	\\
		   \; &= \; T_{n} - T_{0}
	\end{align*}
	
	\paragraph{Problem}
	Calculate the given sum, to $n$ terms
	\begin{equation*}
		S_n  \;=\;  1(0!) + 3(1!) + 7(2!) + 13(3!) + 21(4!) + \dots + T_{n-1}
	\end{equation*}
	where $T_{k}$ is the $k^{\text{th}}$ term of the given series.
	\paragraph{Solution}
	Observe
	\begin{align*}
		T_k  \; &= \;  (k^2 + k + 1)\cdot k!
	\end{align*}
	Consider the definition
	\begin{align*}
		F_{k}  \; &= \;  k\cdot k!
	\end{align*}
	Note that $n! = n\cdot (n-1)!$
	\begin{align*}
		F_{k+1}  \; &= \; (k + 1)\cdot (k + 1)! \\
			 \; &= \; (k + 1)\cdot (k + 1)\cdot k! \\
			 \; &= \; (k^2 + 2k + 1)\cdot k! \\
			 \; &= \; (k^2 + k + 1)\cdot k! + k\cdot k! \\
			 \; &= \; T_k + F_k \\
		\Aboxed{T_k \; &= \;  F_{k+1} - F_{k}}
	\end{align*}
	The required sum $S_n$ is thus is given by
	\begin{align*}
		S_n  	\; &= \; \sum_{k=0}^{n-1} T_k \\
		 	\; &= \; \sum_{k=0}^{n-1} (F_{k+1} - F_{k}) \\
			\; &= \; F_n - F_0 \\
			\; &= \; n\cdot n! - 0\cdot 0! \\\\
		\Aboxed{S_n	\; &= \;  n\cdot n!}
	\end{align*}
	Hence, it is obvious that $S_{4000} = 4000\cdot 4000!$
\end{document}
