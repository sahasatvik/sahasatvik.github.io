\documentclass[11pt,reqno]{article}
\usepackage{geometry}
\geometry{letterpaper}

\usepackage{graphicx}
\usepackage{amssymb}
\usepackage{amsmath}
\usepackage{tikz}
\usetikzlibrary{arrows,calc}

\DeclareGraphicsRule{.tif}{png}{.png}{`convert #1 `dirname #1`/`basename #1 .tif`.png}

\title{The Range of a Projctile}
\author{Satvik Saha}
\date{}

\begin{document}
\maketitle

\begin{center}
\begin{tikzpicture}[scale=1.4]
	\coordinate [label=right:$x$]		(X)		at	(8,0);
	\coordinate [label=above:$y$]		(Y)		at	(0,4);
	\coordinate 						(A)		at	(0,2);
	\coordinate							(B)		at	(7,0);
	
	\draw[<->] (X) -| (Y);
	\draw[<->, >=stealth, line width=0.2mm] (-1mm,0) -- ($ (A) - (1mm,0)$)
											node[midway, left] {$y_0$};
	\draw[<->, >=stealth, line width=0.2mm] (0,-1mm) -- ($ (B) - (0,1mm)$)
											node[midway, below] {$R$};
											
	\draw[dotted] (A) -- ($(A) + (X)$);
	\draw[-latex, thick] (A) -- ++ (38:2) node[midway, above, sloped] {$\vec{v}_0$};
	\draw[-latex] (A) ++ (0:4mm) arc (0:38:4mm);
	\node at ($ (A) + (6mm, 2mm)$) {$\theta$};
	
	\draw[dotted, very thick, blue] (A) .. controls ($(A) + (38:4)$) and ($(A) + (10:5)$) .. (B);
	
	
\end{tikzpicture}
\end{center}

Consider the equations of motion of a projectile, launched at an elevation $\theta$ from a height $y_0$, experiencing uniform acceleration $-g$ along the $y$-axis.

\begin{align}
	x(t)	\;&=\;	v_0 \cos\theta									\\
	y(t)	\;&=\;	y_0 \;+\; v_0 t \sin\theta \;-\;\frac{1}{2}gt^2
\end{align}

When the projectile hits the ground, we see that $y(t) = 0$. Let this time be $t{flight}$ and the corresponding horizontal displacement be $R$.

\begin{align*}
	0		\;&=\;	y_0 \;+\; v_0 \sin\theta \;-\;\frac{1}{2}gt^2	\\
	t_{flight}	\;&=\;	\frac{1}{g}(v_0 \sin\theta \;+\; \sqrt{v_0^2\sin^2\theta + 2gy_0}) \\\\
	R		\;&=\;	\frac{1}{g}(v_0 \cos\theta)(v_0 \sin\theta \;+\; \sqrt{v_0^2\sin^2\theta + 2gy_0})
	\tag{3}
\end{align*}

For $R = R_{max}$, we have $\frac{d}{d\theta}R = 0$

\end{document}  