\documentclass[11pt,reqno]{article}
\usepackage{geometry}
\geometry{letterpaper}

\usepackage{amssymb}
\usepackage{mathtools}

\title{Variable force on a Moving Body}
\author{Satvik Saha}
\date{}

\begin{document}
\maketitle

\paragraph{Question}
A body of mass $2.5 \text{ kg}$, is at rest at the position $x = 4$ at $t = 0$.
It experiences a force given by the expression $\vec{F} = -10x + 20 \hat{i}$. Express the position of the
body as a function of time.

\paragraph{Solution}
We have $\vec{F} = m\vec{a} = -10x + 20 \hat{i}$ and $m = 2.5$. Thus
\begin{align*}
\vec{a} \;&=\; -4x + 8 \;\hat{i}
\end{align*}

Changing notation to scalars and replacing $a$ with $x''$
\begin{align*}
\boxed{x'' \,+\, 4x \,-\, 8 \;=\; 0} \tag{1}
\end{align*}

This is a second order non-homogeneous double differential equation.

Let the solutions to $(1)$ be of the form 
\begin{align*}
x(t) \;=\; x_C(t) + x_P(t) \tag{2}
\end{align*}

where $x_C(t)$ is the solution to $x'' + 4x = 0$. \\

Make the substitution $x = e^{\lambda t}$, for some constant $\lambda$. Thus
\begin{align*}
\frac{d^2}{dt^2}e^{\lambda t} + 4e^{\lambda t} \;&=\; 0    \\
\lambda^2 e^{\lambda t} + 4e^{\lambda t} \;&=\; 0    \\
\lambda^2 + 4 \;&=\; 0    \\
\lambda \;&=\; \pm 2i
\end{align*}

We have $x_C(t) = c_1x_1 + c_2x_2$ for each value of $x$ corresponding with a value of 
$\lambda$, *ie*, $x_1 = e^{2it}$ and $x_2 = e^{-2it}$. Thus

\begin{align*}
x_C(t) \;&=\; c_1 e^{2it} + c_2 e^{-2it}
\end{align*}

Applying Euler's Formula, $e^{ix} = \cos x + i\sin x$, we have
\begin{align*}
x_C(t) \;&=\; c_1(\cos(2t) + i sin(2t)) + c_2(\cos(2t) - i\sin(2t)) \\
\;&=\; (c_1 + c_2)\cos(2t) + i(c_1 - c_2)\sin(2t)     \\
\;&=\; k_1\cos(2t) + k_2\sin(2t) \tag{3}
\end{align*}

Let $x_P(t) = a_1 \implies x_p''(t) = 0$ for some constant $a_1$. Substituting this in $(1)$ gives
\begin{align*}
x_P''(t) + 4x_P(t) \;&=\; 8      \\
x_P(t) \;&=\; 2 \tag{4}
\end{align*}

Thus, from $(1)$, $(3)$ and $(4)$
\begin{align*}
\boxed{x(t) \;=\; k_1\cos(2t) \,+\, k_2\sin(2t) \,+\, 2} \tag{5}
\end{align*}

Note that we know that at $t = 0$, $x = 4$ and $v = x' = 0$. Thus
\begin{align*}
x(0) \;&=\; k_1 + 2 = 4  \\
k_1  \;&=\; 2
\end{align*}

And
\begin{align*}
x'(t) \;&=\; -2k_1\sin(2t) + 2k_2\cos(2t)    \\
x'(0) \;&=\; 2k_2 = 0                        \\
k_2   \;&=\; 0
\end{align*}

Finally, from $(5)$, we can write
\begin{align*}
\boxed{x(t) \;=\; 2\cos(2t) \,+\, 2}
\end{align*}

\end{document}
